\documentclass[titlepage]{article}
\usepackage{loubookman}
\usepackage{loustandard}
\usepackage{graphics}
\usepackage{amsmath}
\usepackage{loumath}
\usepackage{codedocn}

%\newcommand{\mypath}{/cs/mitacs1/lou/Bonsai/Doc/Dylp.Coin091002}
\input{dylpabsdir}
\newcommand{\figures}{\mypath/Figures}

\newsavebox{\tmpbox}

\newcommand{\bonsai}{\textbf{bonsai}\xspace}
\newcommand{\glpk}{GLPK\xspace}
\newcommand{\consys}{\textsc{consys}\xspace}
\newcommand{\dylp}{\textsc{dylp}\xspace}
\newcommand{\Dylp}{\textsc{Dylp}\xspace}
\newcommand{\bonsaiG}{\textbf{bonsaiG}\xspace}
\newcommand{\coin}{\textsc{Coin-OR}\xspace}


\newcommand{\netlib}{Netlib\xspace}
\newcommand{\miplib}{MIPLIB\xspace}

\renewcommand{\textfraction}{.1}
\renewcommand{\topfraction}{.9}
\renewcommand{\bottomfraction}{\topfraction}

\makeatletter

% numberline amounts to \hbox to \@tempdima {box contents}. This change
% will set the section number (#1) right-justified, with a bit more space
% before the section title. See texmf-dist/tex/latex/base/latex.ltx.

\renewcommand{\numberline}[1]{\hb@xt@\@tempdima{\hfil#1\hspace{1.5ex}}}

% In order to accommodate the extra space for numberline, we have to go in
% and change the space for l@section and l@subsection. (Don't need
% l@subsubsection so far, but it's here if needed.) See texmf-dist/tex/latex/
% base/article.cls and Sec. 2.3.2 in the Latex Companion, 2e.

\renewcommand*{\l@section}[2]{%
  \ifnum \c@tocdepth >\z@
    \addpenalty\@secpenalty
    \addvspace{1.0em \@plus\p@}%
    \setlength\@tempdima{3.5em}%
    \begingroup
      \parindent \z@ \rightskip \@pnumwidth
      \parfillskip -\@pnumwidth
      \leavevmode \bfseries
      \advance\leftskip\@tempdima
      \hskip -\leftskip
      #1\nobreak\hfil \nobreak\hb@xt@\@pnumwidth{\hss #2}\par
    \endgroup
  \fi}
\renewcommand*{\l@subsection}{\@dottedtocline{2}{3.5em}{3.5em}}
\renewcommand*{\l@subsubsection}{\@dottedtocline{3}{3.8em}{3.2em}}
\makeatother


\title{\bfseries \dylp: a dynamic LP code}
\author{Lou Hafer}
\date{October, 2009}


\begin{document}

\maketitle

\thispagestyle{empty}
\vspace*{\fill}

\noindent
Copyright (C) 2005, 2006, 2007, 2008, 2009 Lou Hafer

\noindent
An earlier version of this document is available as SFU-CMPT TR 2005-18.

\noindent
All rights reserved. This documentation is made available under the terms of
the Eclipse Public License v1.0 which accompanies this distribution.
A copy of the EPL v1.0 can also be obtained from the URL
\texttt{http://www.eclipse.org/legal/epl-v10.html}

\begin{abstract}\noindent
\dylp is a full implementation of the dynamic simplex algorithm for
linear programming.
Dynamic simplex attempts to maintain a reduced active constraint system
by regularly
purging loose constraints and variables with unfavourable reduced costs, and
adding violated constraints and variables with favourable reduced costs.
In abstract, the code alternates between primal and dual simplex algorithms,
using dual simplex to reoptimise after updating the constraint set and primal
simplex to reoptimise after updating the variable set.
\end{abstract}

\tableofcontents

\pagebreak

\listoffigures
\listoftables

\include{intro}

\section{Notation}
\label{sec:Notation}

\dylp works naturally with the minimisation problem
\begin{equation}
\begin{split}
\min \enspace  cx & \\
      Ax & \leq b \\
l \leq x & \leq u
\end{split} \label{Eqn:BoundedPrimal}
\end{equation}
Add slack variables $s$ and partition $\begin{bmatrix} A & I \end{bmatrix}$
into basic and nonbasic portions as
\begin{equation*}
\begin{split}
\begin{bmatrix} B & N \end{bmatrix} =
\left[
\begin{array}{cc|cc}
B^t & 0 & N^t & I^t \\
B^l & I^l & N^l & 0
\end{array}
\right]
\end{split}
\end{equation*}
with corresponding partitions
\begin{math}
\trans{\begin{bmatrix} x^B & s^B & x^N & s^N \end{bmatrix}}
\end{math}
for $x$, $s$, and
\begin{math}
\trans{\begin{bmatrix} b^t & b^l\end{bmatrix}}
\end{math}
for $b$.
The objective $c$ is augmented with 0's in the columns corresponding to the
slack variables, and partitioned as
\begin{math}
\begin{bmatrix} c^B & 0 & c^N & 0 \end{bmatrix}
\end{math}.
The basis inverse will be 
\begin{equation}
\inv{B} = \begin{bmatrix}
	     \inv[0]{(B^t)} & 0 \\
	     -B^l\inv[0]{(B^t)} & I^l
	    \end{bmatrix}
	    \label{Eqn:PrimalBasisInverse}
\end{equation}
We then have
\begin{equation}
\begin{split}
\begin{bmatrix} x^B \\ s^B \end{bmatrix} & = \,
\inv{B}b - \inv{B} N \begin{bmatrix} x^N \\ s^N \end{bmatrix} \\
%
& = \begin{bmatrix}
      \inv[0]{(B^t)} b^t \\ b^l - B^l\inv[0]{(B^t)} b^t
    \end{bmatrix} -
    \begin{bmatrix}
      \inv[0]{(B^t)} N^t & \inv[0]{(B^t)} \\
      N^l - B^l\inv[0]{(B^t)}N^t & -B^l\inv[0]{(B^t)}
    \end{bmatrix}
    \begin{bmatrix} x^N \\ s^N \end{bmatrix}
\end{split} \label{Eqn:PrimalBasicVars}
\end{equation}
and
\begin{equation}
\begin{split}
z & = \begin{bmatrix} c^B & 0 \end{bmatrix}
      \trans{\begin{bmatrix} x^B & s^B \end{bmatrix}} +
      \begin{bmatrix} c^N & 0 \end{bmatrix}
      \trans{\begin{bmatrix} x^N & s^N \end{bmatrix}} \\
  & = \begin{bmatrix} c^B & 0 \end{bmatrix}\inv{B}b +
      \left(
        \begin{bmatrix} c^N & 0 \end{bmatrix} -
        \begin{bmatrix} c^B & 0 \end{bmatrix} \inv{B} N
      \right)
      \trans{\begin{bmatrix} x^N & s^N \end{bmatrix}} \\
  & = c^B \inv[0]{(B^t)} b^t +
      \begin{bmatrix}
	c^N - c^B \inv[0]{(B^t)} N^t & -c^B\inv[0]{(B^t)} \end{bmatrix}
      \trans{\begin{bmatrix} x^N & s^N \end{bmatrix}}
\end{split} \label{Eqn:PrimalObj}
\end{equation}
The quantities
$\trans{\begin{bmatrix} x^B & s^B \end{bmatrix}} =
\overline{b} = \inv{B}b$ are the values of the basic
variables, the quantities
$y = \begin{bmatrix} c^B & 0 \end{bmatrix}\inv{B}$ are the dual
variables, and the quantities
$\overline{c} = \left(
		  \begin{bmatrix} c^N & 0 \end{bmatrix} -
		  \begin{bmatrix} c^B & 0 \end{bmatrix} \inv{B} N
		\right)$
are the reduced costs.
A row or column of $\inv{B}N$ (as appropriate to the context) will be
denoted $\overline{a}_k$ (the single subscript distinguishes it from an
individual element $\overline{a}_{ij}$).
A row or column of $\inv{B}$ (as appropriate to the context) will be
denoted $\beta_k$.
When discussing pivot selection calculations, $\Delta_j$ will be the
change in nonbasic variable $x_j$ or $s_j$.

The dual problem is formed by first converting \eqref{Eqn:BoundedPrimal} to
$\max\: -cx$, giving
\begin{equation*}
\begin{split}
\min \enspace & yb \\
        & y A \geq -c \\
        & y \geq 0
\end{split}
\end{equation*}
Add surplus variables $\sigma$ and partition
$\trans{\begin{bmatrix}A & -I \end{bmatrix}}$
into basic and nonbasic portions as
\begin{equation*}
\addtolength{\extrarowheight}{3pt}
\begin{bmatrix} \mathcal{B} \\ \mathcal{N} \end{bmatrix} = 
\left[
\begin{array}{cc}
0  & -I^\mathcal{B} \\
B^t & N^t \\
\hline
-I^\mathcal{N} & 0 \\
B^l & N^l
\end{array}
\right]
\end{equation*}
with corresponding partitions
\begin{math}
\begin{bmatrix}
  \sigma^\mathcal{B} & y^\mathcal{B} & \sigma^\mathcal{N} & y^\mathcal{N}
\end{bmatrix}
\end{math}
for $y$, $\sigma$, and
\begin{math}
\begin{bmatrix} c^B & c^N \end{bmatrix}
\end{math}
for $c$.
The right-hand side $b$ is augmented with 0's in the rows corresponding to the
surplus variables and partitioned as
\begin{math}
\trans{\begin{bmatrix} 0 & b^t & 0 & b^l \end{bmatrix}}
\end{math}.
The basis inverse will be 
\begin{equation*}
\inv[-2]{\mathcal{B}} =
\begin{bmatrix}
  \inv[0]{(B^t)} N^t & \inv[0]{(B^t)} \\
  -I^\mathcal{B} & 0
\end{bmatrix}.
\end{equation*}
We then have
\begin{equation}
\begin{split}
\begin{bmatrix} \sigma^\mathcal{B} & y^\mathcal{B} \end{bmatrix} & =
    (-c)\inv[-2]{\mathcal{B}} -
    \begin{bmatrix} \sigma^\mathcal{N} & y^\mathcal{N} \end{bmatrix}
    \mathcal{N}\inv[-2]{\mathcal{B}} \\
& = \begin{bmatrix}
      c^N - c^B \inv[0]{(B^t)} N^t & -c^B \inv[0]{(B^t)}
    \end{bmatrix} -
    \begin{bmatrix} \sigma^\mathcal{N} & y^\mathcal{N} \end{bmatrix}
    \begin{bmatrix}
      -\inv[0]{(B^t)} N^t & -\inv[0]{(B^t)} \\
      B^l\inv[0]{(B^t)} N^t - N^l & B^l\inv[0]{(B^t)}
    \end{bmatrix} \label{Eqn:DualBasicVars}
\end{split}
\end{equation}
and
\begin{equation}
\begin{split}
z & = \begin{bmatrix} \sigma^\mathcal{B} & y^\mathcal{B} \end{bmatrix}
      \trans{\begin{bmatrix} 0 & b^t \end{bmatrix}} +
      \begin{bmatrix} \sigma^\mathcal{N} & y^\mathcal{N} \end{bmatrix}
      \trans{\begin{bmatrix} 0 & b^l \end{bmatrix}} \\
  & = (-c)\inv[-2]{\mathcal{B}} b^\mathcal{B} +
      \begin{bmatrix} \sigma^\mathcal{N} & y^\mathcal{N} \end{bmatrix}
      (b^\mathcal{N} - \mathcal{N}\inv[-2]{\mathcal{B}} b^\mathcal{B}) \\
  & = -c^B \inv[0]{(B^t)} b^t +
      \begin{bmatrix} \sigma^\mathcal{N} & y^\mathcal{N} \end{bmatrix}
      \begin{bmatrix}
	\inv[0]{(B^t)} b^t \\ b^l - B^l\inv[0]{(B^t)} b^t
      \end{bmatrix}
\end{split} \label{Eqn:DualObj}
\end{equation}
When discussing pivot selection calculations, $\delta_j$ will be the
change in nonbasic dual variable $y_j$ or $\sigma_j$.

There are several points to note about the relationship between primal and dual
simplex in the \dylp implementation.

First, \dylp does not solve $\max \; -cx$ as a surrogate for $\min \, cx$.
It minimises $cx$ directly by algorithmic design.
Hence the dual variables $y = c^B \inv{B}$ have the wrong sign for the dual
problem, and are calculated solely as a convenience.
The dual algorithm actually works with the reduced costs
$\overline{c}^N = c^N - c^B \inv{B} N$, which are the correct dual variable
values
(compare \eqnref{Eqn:PrimalObj} with \eqnref{Eqn:DualBasicVars}).

Second, because primal simplex provides
$\inv{B} N = - \mathcal{N} \inv[-2]{\mathcal{B}}$
(compare \eqnref{Eqn:PrimalBasicVars} with \eqnref{Eqn:DualBasicVars}),
the relevant calculation when determining the leaving dual variable is
$\overline{c}_k + \overline{a}_{ik} \delta_i$, rather than
$ \overline{c}_k - \overline{a}_{ik} \delta_i$.

Throughout the remainder of the report, let $e_k \in R^d$ be a row or
column vector of appropriate
dimension (as determined by the context), with a 1 in
position $k$ and 0's in all other positions.

\include{updateformulae}
\section{Pricing Algorithms}
\label{sec:PricingAlgorithms}

\subsection{Projected Steepest Edge Pricing}
\label{sec:PSEPricing}

The primal simplex algorithm in \dylp uses projected steepest edge (PSE)
pricing;
the algorithm used is described as dynamic projected steepest edge
(`dynamic') in Forrest and Goldfarb \cite{For92}.

To understand the operation of projected steepest edge (PSE) pricing, it
will be helpful to start with the definition of a direction of motion.
The values of the basic and nonbasic variables can be expressed as
\begin{equation} \label{eqn:allPrimalDirs}
\begin{bmatrix} x^B \\ x^N \end{bmatrix} =
  \begin{bmatrix} b \\ l/u \end{bmatrix} -
  \begin{bmatrix} \inv{B} A^N \\ -I \end{bmatrix} \Delta
\end{equation}
where $l/u$ is intended to indicate use of the lower or upper bound as
appropriate for the particular nonbasic variable.
When a given nonbasic variable $x_j$ is moved by an amount $\Delta_j$, the
values of $x$ will change as
\begin{equation} \label{eqn:onePrimalDir}
  -\begin{bmatrix} \inv{B} a_j \\ -e_j \end{bmatrix} \Delta_j =
  -\begin{bmatrix} \overline{a}_j \\ -e_j \end{bmatrix} \Delta_j =
  \eta_j\Delta_j
\end{equation}
The vector $\eta_j$ is the direction of motion as $x_j$ is changed;
alternatively, it is the edge of the polyhedron which is traversed as $x_j$ is
changed.
Let $\gamma_j = \norm{\eta_j}$ be the norm of $\eta_j$.

For pricing, it can be immediately seen that
$c\eta_j = c_j - c^B \, \overline{a}_j$ is the reduced cost $\overline{c}_j$.
Dantzig pricing chooses an entering variable $x_j$ such
that $\overline{c}_j$ has appropriate sign and the largest magnitude over all
reduced costs, but it can be misled by differences in scaling from
one column to the next.
Steepest edge (SE) pricing scales $\overline{c}_j$ by $\gamma_j$, choosing
an entering variable $x_j$ with $\overline{c}_j$ of appropriate sign
and the largest $\displaystyle \abs{\frac{c \eta_j}{\norm{\eta_j}}}$,
effectively
calculating the change in objective value over a unit vector in the direction
of motion.
This gives a uniform pricing comparison, using the slope of the edge.

Projected steepest edge (PSE) pricing uses `projected' column
norms which are calculated using a vector $\tilde{\eta}_j$ which contains only
the components of $\eta_j$ included in a reference frame.
Initially, this reference frame contains only the nonbasic variables, so that
$\tilde{\gamma}_j = 1$ for all $x_j \in x^N$.
In order to avoid calculating $\tilde{\gamma}_j$ from scratch each time a
column must be priced, the norms are iteratively updated.

To derive the update formul\ae{} for $\tilde{\gamma}_j$, it is useful to start
with the update formul\ae{} for the full vector $\eta_j$.
As mentioned in \secref{sec:DualUpdates},
for $x_i$ leaving basis position $k$ and
$x_j$ entering, $B\inv[0]{(B')} = I + a_i(\beta')_k - a_j(\beta')_k$.
Taking this one step further, 
$\inv[0]{(B')} = \inv{B} + \overline{a}_i(\beta')_k - \overline{a}_j(\beta')_k$.
Then for an arbitrary column $a_p$,
\begin{align}
\inv[0]{(B')} a_p & = \inv{B} a_p + \overline{a}_i(\beta')_k a_p -
	\overline{a}_j(\beta')_k a_p \notag \\
\overline{a}'_p & = \overline{a}_p +
	e_k ( \frac{\overline{a}_{kp}}{\overline{a}_{kj}} ) -
	\overline{a}_j ( \frac{\overline{a}_{kp}}{\overline{a}_{kj}} )
	\label{Eqn:abarupdate}
\end{align}
(recalling that $(\beta')_k = \beta_k/\overline{a}_{kj}$).

To see that (\ref{Eqn:abarupdate}) amounts to
$\eta'_p = \eta_p - \eta_j( \dfrac{\overline{a}_{kp}}{\overline{a}_{kj}} )$,
it's helpful to expand the vectors:
\begin{equation*}
\overline{a}'_p =
  \begin{bmatrix} \overline{a}_{1p} \\ \vdots \\
		  \overline{a}_{kp} \\ \vdots \\
		  \overline{a}_{mp} \end{bmatrix} +
  \begin{bmatrix} 0 \\ \vdots \\
		  1 \\ \vdots \\
		  0 \end{bmatrix}\frac{\overline{a}_{kp}}{\overline{a}_{kj}} -
  \begin{bmatrix} \overline{a}_{1j} \\ \vdots \\
		  \overline{a}_{kj} \\ \vdots \\
		  \overline{a}_{mj} \end{bmatrix}
		  \frac{\overline{a}_{kp}}{\overline{a}_{kj}} .
\end{equation*}
With a little thought, it can be seen that the middle term represents one
half of the
permutation which moves $x_j$ into the basic partition of $\eta'_j$.
(The other half moves $x_i$ into the nonbasic partition).
When updating $\eta_i$, the update formula can be collapsed to
$\eta'_i = - \eta_j/\overline{a}_{kj}$, since $\overline{a}_{ki} = 1$.
Summarising, the update formul\ae{} for the edge directions $\eta_j$ are
\begin{equation}
\begin{split}
\eta'_p & = \eta_p - \eta_j( \frac{\overline{a}_{kp}}{\overline{a}_{kj}} ),
\qquad p \neq i \\
\eta'_i & = - \eta_j/\overline{a}_{kj}. \label{Eqn:etaupdate}
\end{split}
\end{equation}

In fact, the code actually stores and updates $\gamma_j^{\,2}$.
With (\ref{Eqn:etaupdate}) in hand, derivation of the update formul\ae{}
are straightforward:
\begin{align}
\begin{split}
(\gamma^{\,\prime}_p)^2 & = \eta'_p \cdot \eta'_p \\
  & = (\eta_p - \eta_j( \frac{\overline{a}_{kp}}{\overline{a}_{kj}} )) \cdot
      (\eta_p - \eta_j( \frac{\overline{a}_{kp}}{\overline{a}_{kj}} )) \\
  & = \eta_p \cdot \eta_p -
      2( \frac{\overline{a}_{kp}}{\overline{a}_{kj}} )\eta_j \cdot \eta_p +
      ( \frac{\overline{a}_{kp}}{\overline{a}_{kj}} )^2 \eta_j \cdot \eta_j \\
  & = \gamma_p^{\,2} -
      2( \frac{\overline{a}_{kp}}{\overline{a}_{kj}} )
      \begin{bmatrix} \overline{a}^T_j & e^T_j \end{bmatrix}
      \begin{bmatrix} \overline{a}_p \\ e_p \end{bmatrix} +
      ( \frac{\overline{a}_{kp}}{\overline{a}_{kj}} )^2 \gamma_j^{\,2} \\
  & = \gamma_p^{\,2} -
      2( \frac{\overline{a}_{kp}}{\overline{a}_{kj}} )
      (\overline{a}^T_j \inv{B}) a_p +
      ( \frac{\overline{a}_{kp}}{\overline{a}_{kj}} )^2 \gamma_j^{\,2}
      \label{Eqn:gammapupdate}
\end{split} \\[.5ex]
\begin{split}
(\gamma^{\,\prime}_i)^2 & = \eta'_i \cdot \eta'_i \\
  & = \eta_j/\overline{a}_{kj} \cdot \eta_j/\overline{a}_{kj} \\
  & = \gamma_j^{\,2}/\overline{a}^2_{kj}  \label{Eqn:gammaiupdate}
\end{split}
\end{align}

Equations (\ref{Eqn:etaupdate}) can be used directly to update the
$\tilde{\eta}_j$.
To adapt (\ref{Eqn:gammapupdate}) and (\ref{Eqn:gammaiupdate}) for the
$\tilde{\gamma}_j$,
a little algebra should serve to see that it's sufficient to substitute
$\tilde{a}_j$ in (\ref{Eqn:gammapupdate}), as well as using
$\tilde{\gamma}_p$ and $\tilde{\gamma}_j$.

It is straightforward to observe that when equations (\ref{Eqn:etaupdate})
are premultiplied by $c$, they can be used to update the reduced costs as
\begin{align*}
\begin{split}
\overline{c}'_p & = \overline{c}_p -
	\overline{c}_j( \frac{\overline{a}_{kp}}{\overline{a}_{kj}} )
\qquad p \neq i \\
\overline{c}'_i & = - \overline{c}_j/\overline{a}_{kj}.
\end{split}
\end{align*}

\subsection{Dual Steepest Edge Pricing}
\label{sec:DSEPricing}

The dual simplex in \dylp uses dual steepest edge (DSE) pricing; the algorithm
used is described as dual algorithm 1 (`steepest 1') in
Forrest and Goldfarb \cite{For92}.

The values $\overline{b} = \inv{B}b$ are the reduced costs of the nonbasic
dual variables.
Analogous to Dantzig pricing in the primal case, one can choose a entering
dual variable $y_i$ such that $\overline{b}_i$ has appropriate sign and the
largest magnitude over all reduced costs, but there is the same problem with
scaling.
The version of dual steepest edge (DSE) pricing implemented in \dylp
scales $\overline{b}_i = \beta_i b$ by $\rho_i = \norm{\beta_i}$,
choosing
a leaving variable $x_i$ with $\overline{b}_i$ of appropriate sign
and the largest $\displaystyle \abs{\frac{\beta_i b}{\norm{\beta_i}}}$,
effectively
calculating the change in the dual objective value over a unit vector in
the dual direction of motion in the space of the dual variables.
This gives a uniform pricing comparison, using the slope of the dual edge.

In the next few paragraphs, an alternative motivation of the algorithm is
presented which (perhaps) clarifies the relationship between dual
algorithm 1 and dual algorithm 2 in that paper%
\footnote{Those who have
read \cite{For92} are warned that the author's notation is in no way compatible
with that of Forrest and Goldfarb.}.

To see how DSE operates within the context of the revised primal simplex
tableau, we can refer back to equations \eqnref{Eqn:DualBasicVars} and
\eqnref{Eqn:DualObj} from \secref{sec:Notation}, repeated here:
\begin{equation}
\begin{split}
\begin{bmatrix} \sigma^\mathcal{B} & y^\mathcal{B} \end{bmatrix} & =
    (-c)\inv[-2]{\mathcal{B}} -
    \begin{bmatrix} \sigma^\mathcal{N} & y^\mathcal{N} \end{bmatrix}
    \mathcal{N}\inv[-2]{\mathcal{B}} \\
  & = \begin{bmatrix}
	c^N - c^B \inv[0]{(B^t)} N^t & -c^B \inv[0]{(B^t)}
      \end{bmatrix} -
      \begin{bmatrix} \sigma^\mathcal{N} & y^\mathcal{N} \end{bmatrix}
    \begin{bmatrix}
      -\inv[0]{(B^t)} N^t & -\inv[0]{(B^t)} \\
      B^l\inv[0]{(B^t)} N^t - N^l & B^l\inv[0]{(B^t)}
    \end{bmatrix}
\end{split} \tag{\ref{Eqn:DualBasicVars}}
\end{equation}
and
\begin{equation}
\begin{split}
z & = \begin{bmatrix} \sigma^\mathcal{B} & y^\mathcal{B} \end{bmatrix}
      \begin{bmatrix} 0 & b^t \end{bmatrix}^T +
      \begin{bmatrix} \sigma^\mathcal{N} & y^\mathcal{N} \end{bmatrix}
      \begin{bmatrix} 0 & b^l \end{bmatrix}^T \\
  & = (-c)\inv[-2]{\mathcal{B}} b^\mathcal{B} +
      \begin{bmatrix} \sigma^\mathcal{N} & y^\mathcal{N} \end{bmatrix}
      (b^\mathcal{N} - \mathcal{N}\inv[-2]{\mathcal{B}} b^\mathcal{B}) \\
  & = -c^B \inv[0]{(B^t)} b^t +
      \begin{bmatrix} \sigma^\mathcal{N} & y^\mathcal{N} \end{bmatrix}
      \begin{bmatrix}
	 \inv[0]{(B^t)} b^t \\ b^l - B^l\inv[0]{(B^t)} b^t
      \end{bmatrix}
\end{split} \tag{\ref{Eqn:DualObj}}
\end{equation}
Recall that the values of the dual basic variables are the reduced costs of
the primal problem, and the reduced costs of the dual variables are the values
of the primal basic variables (\cf equations \eqnref{Eqn:PrimalBasicVars} and
\eqnref{Eqn:PrimalObj}).

By analogy to the primal pivoting rules, for dual simplex
we want to choose a nonbasic dual variable which will move us in a direction
of steepest descent.
If the nonbasic dual is to increase, its reduced cost must be less than 0 in
order to see a reduction in the dual objective.
This corresponds to the case of a primal variable which will be increased and
driven out of the basis at its lower bound with a positive primal reduced
cost.
If the nonbasic dual is to decrease, its reduced cost must be greater than
0 in order to see a reduction in the dual objective.
This corresponds to the case of a primal variable which will be decreased and
driven out of the basis at its upper bound with a negative primal reduced cost.

The actual direction of motion in the full dual space ($y$ and $\sigma$) is
specified by a row of
\begin{equation*}
\mathcal{N}\inv[-2]{\mathcal{B}} = \begin{bmatrix}
			-\inv[0]{(B^t)} N^t & -\inv[0]{(B^t)} \\
			B^l\inv[0]{(B^t)} N^t - N^l & B^l\inv[0]{(B^t)}
		      \end{bmatrix},
\end{equation*}
a vector which is not readily available in the revised
primal simplex\footnote{%
It's necessary to calculate one such row $\overline{a}_i$ once the entering
dual variable has been selected, but only one.
For the typical problem in which the number of variables greatly
exceeds the number of constraints, the norms of these vectors are expensive to
calculate when initialising the pricing algorithm, and the updates are
expensive.
The algorithm which uses the full dual direction of motion is the one that
Forrest and Goldfarb describe as dual algorithm 2.}.
However, one can make an argument that there's no need to consider the
component of the direction of motion in the subspace of the dual surplus
variables when choosing the entering dual variable.
(More positively, we can take the view that we're only interested in motion
in the polyhedron $\{y \in R^m \mid yA \geq -c, y \geq 0\}$ defined by the
dual variables.)
Changes in the surplus variables cannot affect the objective directly, as
they account for the 0's in the augmented and partitioned $b$ vector.
Algebraically, we can see that the dual basic portion
of $b$, $\begin{bmatrix}0 & b^t \end{bmatrix}^T$, guarantees that there will
never be any direct contribution from the columns
of $\mathcal{N}\inv[-2]{\mathcal{B}}$ involving $N$.
The component of motion in the space of the dual variables $y$ is then simply
the rows $\beta_i$ of $\inv{B}$, which are easily available from the primal
tableau.
(The analogous action in the primal problem --- ignore the component of
$\eta_j$ in the subspace of the primal slack variables --- offers no
computational advantage.)

Given a rationale for taking the rows $\beta_i$ of $\inv{B}$ as the
component of interest in the dual direction of motion, what remains is to work
out the details.
Since we're aiming for a steepest edge algorithm, we'll be interested in
iteratively updating $\norm{\beta_i}^2 = \beta_i \cdot \beta_i$, the square
of the norm of a row $\beta_i$.
Given the update formul\ae{} for $\beta_i$ derived in
\secref{sec:BasisUpdates},
the development of the update formul\ae{} for $\rho_i = \norm{\beta_i}^2$ is
straightforward algebra.
Let $x_i$ be the leaving variable and $x_j$ be the entering variable, and
assume $x_i$ occupies row $k$ of the basis $B$ before the update.
We have
\begin{align}
\begin{alignedat}{2}
\rho'_i & = \rho_i -
	    2\frac{\overline{a}_{lj}}{\overline{a}_{kj}}\beta_i\cdot\beta_k +
	    (\frac{\overline{a}_{lj}}{\overline{a}_{kj}})^2\rho_k &
	    \qquad\qquad i \neq k \\
\rho'_k & = (\frac{1}{\overline{a}_{kj}})^2\rho_k
\end{alignedat}
\end{align}
Since the update will be performed for all rows in the basis, it's worth
calculating the vector $\tau = \inv{B}\beta^T_k$ to obtain all the inner
products $\beta_i \cdot \beta_k$ in one calculation.


\include{perturbed}
\include{antidegenlite}
\include{lpbasis}
\include{accuracy}
\section{Scaling}
\label{sec:Scaling}

\dylp provides the capability for row and column scaling of the original
LP problem.
This section develops the algebra used for scaling and describes
some additional details of the implementation.
The following section (\secref{sec:Solutions}) describes unscaling in the
context of generating solution, ray, and tableau vectors for the client.

Let $R$ be a diagonal matrix used to scale the rows of the LP problem and $S$
be a diagonal matrix used to scale the columns of the LP problem.
The original problem \eqnref{Eqn:BoundedPrimal} is scaled as
\begin{align*}
 \min \:(cS)(\inv{S}x) & \\
     (RAS)(\inv{S}x) & \leq (Rb) \\
     (\inv{S}l) \leq (\inv{S}x) & \leq (\inv{S}u)
\intertext{to produce the scaled problem}
  \min \breve{c}\breve{x} & & \\
     \breve{A}\breve{x} & \leq \breve{b} \\
     \breve{l} \leq \breve{x} & \leq \breve{u} \\
\end{align*}
where $\breve{A} = RAS$, $\breve{b} = Rb$, $\breve{c} = cS$,
$\breve{l} = \inv{S}l$, $\breve{u} = \inv{S}u$, and $\breve{x} = \inv{S}x$.
\dylp then treats the scaled problem as the original problem.

By default, \dylp will calculate scaling matrices $R$ and $S$ and scale the
constraint system unless the coefficients satisfy the conditions
$.5 < \min_{ij} \abs{a_{ij}}$ and $\max_{ij} \abs{a_{ij}} < 2$.
The client can forbid scaling entirely, or supply a pair of vectors that will
be used as the diagonal coefficients of $R$ and $S$.

A few additional details are helpful to understand the implementation.
The first is that \dylp uses row scaling to convert `$\geq$' constraints to
`$\leq$' constraints.
Given a constraint system with `$\geq$' constraints, \dylp will generate
scaling vectors with coefficients of $\pm 1.0$ even if scaling is otherwise
forbidden.
If scaling is active for numerical reasons, the relevant row scaling
coefficients will be negated.

\dylp scales the original constraint system before generating
logical variables.
Nonetheless, it is desirable to maintain a coefficient of 1.0 for each logical.
The row scaling coefficient $r_{ii}$ for constraint $i$ is already determined.
To keep the coefficients of logical variables at $1.0$, the column
scaling factor is chosen to be $1/r_{ii}$ and the column scaling matrix $S$
is conceptually extended to include logical variables.


\section{Generating Solutions, Rays, and Tableau Vectors}
\label{sec:Solutions}

The dynamic simplex algorithm implemented by \dylp introduces some unique
challenges when generating solution values, rays, and tableau vectors.
The client expects an answer that corresponds to the full, unscaled constraint
system.
In addition to the standard calculations associated with unscaling, \dylp must
often synthesize portions of the answer corresponding to inactive constraints
or variables, and position the components of the answer to match the original
constraint system.
This becomes even more interesting when the client is asking for answers in the
context of the dual problem.

\subsection{Solution Vectors}
\label{sec:SolutionVectors}

Calculating the values of the unscaled primal variables is the simplest
request.
We have $\breve{x}^B = \inv{\breve{B}}\breve{b}$.
Then
$\breve{x}^B = \inv[0]{(S^B)}\inv{B}\inv{R} R b = \inv[0]{(S^B)}\inv{B}b$
and
$x^B = S^B \breve{x}^B$.
Recall that the column scaling factor for a logical variable will be the
inverse of the row scaling factor, as explained in \secref{sec:Scaling}.
The unscaled values of the nonbasic primal variables (architectural or logical)
can be read directly from the original unscaled $l$ and $u$ vectors.

There is one subtle point about the column scaling factor for logicals which is
not immediately apparent from the simplified presentation in the previous
paragraphs.
Logical variables can be basic for the row occupied by their associated
constraint~---~the `natural' position.
They can also be basic for some other row~---~an `unnatural' position; this is
achieved by a column permutation in the basis.
The correct column scaling factor, when required, is the one associated with
the natural row.
This is more apparent when the column permutation matrix $P$ is made explicit:
\begin{align*}
\breve{B} & = (R B S^B) P \\
\inv{\breve{B}} & = \inv{P}\inv[0]{(S^B)}\inv{B}\inv{R}
\end{align*}

By definition, inactive architectural variables are nonbasic, so their value is
also easily read from the original unscaled $l$ and $u$ vectors.
By definition, inactive constraints are loose and the corresponding logical
would be basic.
Rather than expand the basis inverse,
the value of the logical for an inactive constraint $i$ is calculated
directly as $a_i x$.

Turning to the dual variables, recall first the observation from
\secref{sec:Notation} that the dual variables $y = c^B \inv{B}$ calculated by
\dylp during primal and dual simplex are in fact the negative of the correct
dual variables, a consequence of implementing the relationship
\begin{align*}
\min cx & & \max \; (-c)x & & \min yb & \\
Ax & \leq b & Ax & \leq b &  yA & \geq (-c)
\end{align*}
by algorithmic design rather than actually negating $c$.
When generating dual variable values to return to the client, there is a
choice: should the dual variables be returned with a sign convention
appropriate for \dylp's $\min cx$ problem, or should they be returned with a
sign convention appropriate for the true dual problem?
For all routines returning values associated with the dual problem, \dylp
allows the client to choose the sign convention.

There are two further details to consider: The canonical relationship assumes
all constraints are `$\leq$' constraints and
all primal constraints are explicit.
In reality, the primal problem presented to \dylp typically contains `$\geq$'
constraints, and bounds on variables are usually handled by the algorithm
rather than stated as explicit constraints.
There are a number of possible implementation choices; \dylp chooses the
following:
\begin{itemize}
  \item
  For the duals associated with `$\geq$' constraints, the sign of the dual
  value returned is always negated to match the `$\geq$' that's actually
  present in the constraint matrix.
  This choice is intended to make it easy for the client to use the dual
  variable values with the coefficient matrix as written.

  \item
  For the duals associated with variables that are nonbasic at their upper
  bound (hence negative in \dylp's $\min$ primal convention), the value is
  negated if the user chooses the true dual sign convention.
  This matches the conversion of the implicit upper bound to an explicit
  `$\leq$' constraint in the dual problem.

  To extend this point to upper and lower bounds, when the client requests
  dual variables using the true dual sign convention, \dylp assumes that
  implicit upper bounds are made explicit as $x_j \leq u_j$ and implicit
  lower bounds are made explicit as $-x_j \leq -l_j$.
\end{itemize}


To calculate the values of the unscaled row dual variables $y$, start with
$\breve{y} = \breve{c}^B \inv{\breve{B}}$.
Then
$\breve{y} = c^B S^B \inv[0]{(S^B)} \inv{B} \inv{R} = c^B \inv{B} \inv{R} =
y \inv{R}$
and
$y = \breve{y} R$.
These values must be negated to be correct for the dual problem.
By definition, inactive constraints are not tight, hence the value of the
associated dual variable is zero.

The values of the column dual variables are the primal reduced costs
$\overline{c}_j$.
Starting from $\breve{\overline{c}}_j$, we have
\begin{align*}
\breve{\overline{c}}_j &
    = \breve{c}_j - \breve{c}^B \inv{\breve{B}}\breve{a}_j \\
  & = c_j S_j - c^B S^B \inv[0]{(S^B)}\inv{B}\inv{R} R a_j S_j \\
  & = (c_j - c^B \inv{B} a_j) S_j \\
  & = \overline{c}_j S_j
\end{align*}
hence $c_j = \breve{\overline{c}}_j\inv{(S_j)}$.
For active variables, the value of $\breve{\overline{c}}_j$ is immediately
available.
For inactive variables, \dylp first calculates $\breve{\overline{c}}_j =
\breve{y}\breve{a}_j$.

\subsection{Tableau Vectors}
\label{sec:TableauVectors}

\Dylp implements routines to return four tableau vectors: rows $\beta_i$ and
columns $\beta_j$ of the basis inverse \inv{B}, and rows $\overline{a}_i$ and
columns $\overline{a}_j$ of the transformed constraint matrix $\inv{B}A$.

Given the scaled basis inverse
$\inv{\breve{B}} = \inv[0]{(S^B)}\inv{B}\inv{R}$, the unscaled column of the
basis inverse corresponding to basic variable $x_j$, basic for row $k$,
will be
\begin{equation} \label{eqn:unscaledBetaj}
\beta_j =  S^B \inv{\breve{B}} R e_k = S^B \breve{\beta}_j r_{kk}
\end{equation}

Because \dylp periodically deactivates loose constraints, it is in general
necessary to synthesize the rows of the basis inverse for these
inactive constraints.
The necessary algebra is shown in \eqnref{Eqn:PrimalBasisInverse} and repeated
here for convenience:
\begin{equation} \tag{\ref{Eqn:PrimalBasisInverse}}
\inv{B} = \begin{bmatrix}
	     \inv[0]{(B^t)} & 0 \\
	     -B^l\inv[0]{(B^t)} & I^l
	  \end{bmatrix}
\end{equation}
Let the partition $B^t$ correspond to $B$ in \eqnref{eqn:unscaledBetaj},
$\beta_j$ to a column of $\inv[0]{(B^t)}$, and let the
partition $B^l$ be the coefficients of basic variables in the inactive
constraints.
By definition, the basic variable for an inactive constraint is the logical
associated with the constraint.
Given the unscaled column $\beta_j$ for the active system
from \eqnref{eqn:unscaledBetaj},
\dylp generates the necessary
coefficients from $-B^l\inv[0]{(B^t)}$ by calculating $-B^l\beta_j$.

For all inactive logical variables (\ie, logical variables for inactive
constraints), and active logical variables basic in the natural position,
$\beta_j = e_j$; this is detected as a special case.

Generating the transformed column $\overline{a}_j = \inv{B}a_j$ follows a
similar pattern, with the added complication that the requested column
may not be active.
Given the scaled transformed column
\begin{equation*}
\breve{\overline{a}}_j = \inv{\breve{B}}\breve{N}e_j =
  \inv[0]{(S^B)}\inv{B}\inv{R} R a_j s_{jj} =
  \inv[0]{(S^B)}\inv{B}  a_j s_{jj}
\end{equation*}
the unscaled column $\overline{a}_j$ will be
\begin{equation} \label{eqn:unscaledAbarj}
S^B \; \breve{\overline{a}}_j (1/s_{jj})
\end{equation}

If there are inactive constraints, the remaining coefficients in the column
must be synthesized.
Using the same notation as above for basis inverse columns, we have
\begin{equation*}
\inv{B} a_j = \begin{bmatrix}
	     \inv[0]{(B^t)} & 0 \\[.4ex]
	     -B^l\inv[0]{(B^t)} & I^l
	  \end{bmatrix}
	  \begin{bmatrix} a^t_j \\[.4ex] a^l_j \end{bmatrix}
        = \begin{bmatrix}
	     \inv[0]{(B^t)} a^t_j \\[.4ex]
	     a^l_j - B^l\inv[0]{(B^t)} a^t_j
	  \end{bmatrix}
\end{equation*}
Given the unscaled column $\overline{a}_j$ for the active system
from \eqnref{eqn:unscaledAbarj},
\dylp generates the necessary
coefficients from $a^l_j - B^l\inv[0]{(B^t)} a^t_j$ by calculating
$a^l_j - B^l\overline{a}_j$.

If the requested column is not active, \dylp first generates the portion of the
column $R a_j$ that matches the active constraints, and then proceeds as
described in the previous paragraphs.
Inactive logical variables will be basic in natural position, hence
$\overline{a}_j = e_j$; this is detected as a special case.
The user is cautioned that active basic variables are \textit{not} handled as a
special case.


The work required to generate a row $\beta_i$ of the basis inverse is
similar to that required for a column $\beta_j$.
Given the scaled basis inverse
$\inv{\breve{B}} = \inv[0]{(S^B)}\inv{B}\inv{R}$, the unscaled row of the
basis inverse corresponding to basic variable $x_j$, basic for row $i$,
will be
\begin{equation*}
\beta_i = e_i S^B \inv{\breve{B}} R = s_{jj} \breve{\beta}_i R
\end{equation*}
If the requested row is not active, the basis must be extended as outlined in
previous paragraphs.
\Dylp creates the partially scaled row vector $e_i B^l S^B$, calculates an
intermediate vector $e_i B^l S^B \inv[0]{(\breve{B}^t)}$ and then completes the
calculation by postmultiplying by $R$ to remove the row scaling still present
in \inv[0]{(\breve{B}^t)}.

Regrettably, there's no easy way to calculate $\overline{a}_i$, a row of
the transformed matrix $\inv{B}A$.
\Dylp implements this operation as $\overline{a}_i = \beta_i A$.
The calculation is performed entirely in the original unscaled system.


\subsection{Rays}
\label{sec:Rays}

In several aspects, rays prove to be the most challenging of the three
solution components.
Careful attention to sign reversals is required for both primal and dual rays,
and the virtual nature of the dual problem adds yet another layer to the
challenge of generating coefficients for inactive portions of the constraint
system.
The routines implemented in \dylp will return all rays emanating from the
current extreme point up to a limit specified by the client.

For a primal ray $r$, it must be the case that $c r < 0$,
and $a_i r \leq 0$ for a `$\leq$' constraint,
$a_i r \geq 0$ for a `$\geq$' constraint, and
$a_i r = 0$ for range constraints and equalities.
The task of identifying a ray is easy; indeed, it's a simplified version of the
algorithm used to select the leaving variable in a primal pivot, where the only
concern is that no basic variable is driven to bound.
Getting the sign right requires a bit of thought, however.
The relevant mathematics is shown in equations \eqnref{eqn:allPrimalDirs} and
\eqnref{eqn:onePrimalDir}; \eqnref{eqn:onePrimalDir} is repeated here for
convenience:
\begin{equation} \tag{\ref{eqn:onePrimalDir}}
  -\begin{bmatrix} B^{\,-1}a_j \\ -e_j \end{bmatrix} \Delta_j =
  -\begin{bmatrix} \overline{a}_j \\ -e_j \end{bmatrix} \Delta_j =
  \eta_j\Delta_j
\end{equation}
As can be immediately seen, $r$ is precisely
$\eta_j = -\trans{\begin{bmatrix} \overline{a}_j & -e_j \end{bmatrix}}$;
the operations
required for unscaling have been discussed in \secref{sec:TableauVectors}.

In addition to the obvious negation required to produce $\eta_j$, there are
two other possible negations to consider.
\begin{itemize}
  \item
  If the nonbasic variable $x_j$ is actually decreasing from its upper bound
  $u_j$, the ray must be negated to compensate.

  \item
  If the nonbasic variable is a logical $s_i$ associated with a `$\geq$'
  constraint in the original system, \dylp's input transformations have
  converted
  $a_i x \geq b_i \Rightarrow (-a_i)x \leq (-b_i)
    \Rightarrow (-a_i)x + s_i = (-b_i)$
  for $0 \leq s_i \leq \infty$.
  The inverse converts
  $(-a_i)x + s_i = (-b_i) \Rightarrow a_i x + s'_i = b_i
    \Rightarrow a_i x \geq b_i$
  for $-\infty \leq s_i' \leq 0$.
  What appears to be a slack variable increasing from its lower bound is
  actually a surplus variable decreasing from its upper bound; accordingly, the
  ray must be negated.
\end{itemize}

There's no need to synthesize the components of $\eta_j$ that would be
associated with inactive constraints.
By definition, the basic variable for an inactive constraint is the associated
logical.
The ray $r$ contains only the components associated with architectural
variables.

For a dual ray $r$, it must be the case that $rb < 0$ and $rA \geq 0$ for the
true dual problem.
Unfortunately, as outlined in \secref{sec:SolutionVectors}, the primal~--~dual
transform implemented in \dylp does not match the ideal, and this introduces
complications.
Neither mathematical test is guaranteed to work unless the dual variables
associated with tight implicit bound constraints (\ie, nonbasic primal
variables) are handled explicitly.

As with primal rays, the task of identifying a dual ray is easy, a simplified
version of the algorithm used to select the leaving dual variable in a dual
pivot.
The only concern is that no dual basic variable be driven to bound.
Again, it's getting the sign right that requires some thought.

As discussed in \secref{sec:Notation}, the vector $\overline{a}_k$ is the
proper starting point; the initial negation which would normally be required is
built in by $\mathcal{N}\inv[-2]{\mathcal{B}} = -\inv{B}N$.
The operations required for unscaling have been discussed in
\secref{sec:TableauVectors}.
It's necessary to add a coefficient of 1.0 for the nonbasic dual that's
driving the ray.

There are three other sources of negation to consider:
\begin{itemize}
  \item
  If the entering dual is apparently decreasing because it's associated with a
  leaving primal variable that's decreasing to its upper bound (and hence
  must have a
  negative reduced cost when it becomes nonbasic), the ray must be negated
  to compensate.

  \item
  If the ray is derived from a `$\geq$' constraint in the original system,
  the coefficients of the constraint have been negated; this is encoded in the
  row scaling.
  However, as noted for primal rays, the logical must really be interpreted as
  a surplus variable with an upper bound of zero, and if it's basic for this
  row we have the case described in the previous item.
  The ray must be negated.

  \item
  As explained in \secref{sec:SolutionVectors},
  if an individual ray coefficient corresponds to a variable that is nonbasic
  at its upper bound, the ray coefficient must be negated if the client has
  requested the true dual sign convention.
\end{itemize}


\include{startup}
\include{dynamic}
\include{dual}
\include{primal}
\include{varmgmt}
\include{conmgmt}
\include{interface}
\include{statistics}
\include{debug}

\bibliographystyle{louplain}
\bibliography{dylp}

\end{document}
